%% This Beamer template is based on the one found here: https://github.com/sanhacheong/stanford-beamer-presentation, and edited to be used for Stanford ARM Lab

\documentclass[10pt]{beamer}
%\mode<presentation>{}

\usepackage{media9}
\usepackage{amssymb,amsmath,amsthm,enumerate}
\usepackage[utf8]{inputenc}
\usepackage{array}
\usepackage[parfill]{parskip}
\usepackage{graphicx}
\usepackage{caption}
\usepackage{subcaption}
\usepackage{bm}
\usepackage{amsfonts,amscd}
\usepackage[]{units}
\usepackage{listings}
\usepackage{multicol}
\usepackage{multirow}
\usepackage{tcolorbox}
\usepackage{physics}
\usepackage{amsmath}


% Enable colored hyperlinks
\hypersetup{colorlinks=true}

% The following three lines are for crossmarks & checkmarks
\usepackage{pifont}% http://ctan.org/pkg/pifont
\newcommand{\cmark}{\ding{51}}%
\newcommand{\xmark}{\ding{55}}%

% Numbered captions of tables, pictures, etc.
\setbeamertemplate{caption}[numbered]

%\usepackage[superscript,biblabel]{cite}
\usepackage{algorithm2e}
\renewcommand{\thealgocf}{}

% Bibliography settings
\usepackage[style=ieee]{biblatex}
\setbeamertemplate{bibliography item}{\insertbiblabel}
\addbibresource{references.bib}

% Glossary entries
\usepackage[acronym]{glossaries}
\newacronym{ML}{ML}{machine learning}
\newacronym{HRI}{HRI}{human-robot interactions}
\newacronym{RNN}{RNN}{Recurrent Neural Network}
\newacronym{LSTM}{LSTM}{Long Short-Term Memory}


\theoremstyle{remark}
\newtheorem*{remark}{Remark}
\theoremstyle{definition}

\newcommand{\empy}[1]{{\color{darkorange}\emph{#1}}}
\newcommand{\empr}[1]{{\color{cardinalred}\emph{#1}}}
\newcommand{\examplebox}[2]{
\begin{tcolorbox}[colframe=darkcardinal,colback=boxgray,title=#1]
#2
\end{tcolorbox}}

\usetheme{Stanford} 
\def \i  {\item}
\def \ai {\item[] \quad \arrowbullet}
\newcommand \si[1]{\item[] \quad \bulletcolor{#1}}
\def \wi {\item[] \quad $\ \phantom{\Rightarrow}\ $}
\def \bi {\begin{itemize}\item}
\def \ei {\end{itemize}}
\def \be {\begin{equation*}}
\def \ee {\end{equation*}}
\def \bie {$\displaystyle{}
\def \eie {{\ }$}}
\def \bsie {\small$\displaystyle{}
\def \esie {{\ }$}\normalsize\selectfont}
\def \bse {\small\begin{equation*}}
\def \ese {\end{equation*}\normalsize}
\def \bfe {\footnotesize\begin{equation*}}
\def \efe {\end{equation*}\normalsize}
\renewcommand \le[1] {\\ \medskip \lefteqn{\hspace{1cm}#1} \medskip}
\def \bex {\begin{example}}
\def \eex {\end{example}}
\def \bfig {\begin{figure}}
\def \efig {\end{figure}}
\def \btheo {\begin{theorem}}
\def \etheo {\end{theorem}}
\def \bc {\begin{columns}}
\def \ec {\end{columns}}
\def \btab {\begin{tabbing}}
\def \etab {\end{tabbing}\svneg\svneg}
\newcommand \col[1]{\column{#1\linewidth}}
\def\vneg  {\vspace{-5mm}}
\def\lvneg {\vspace{-10mm}}
\def\svneg {\vspace{-2mm}}
\def\tvneg {\vspace{-1mm}}
\def\vpos  {\vspace{5mm}}
\def\lvpos {\vspace{10mm}}
\def\svpos {\vspace{2mm}}
\def\tvpos {\vspace{1mm}}
\def\hneg  {\hspace{-5mm}}
\def\lhneg {\hspace{-10mm}}
\def\shneg {\hspace{-2mm}}
\def\thneg {\hspace{-1mm}}
\def\hpos  {\hspace{5mm}}
\def\lhpos {\hspace{10mm}}
\def\shpos {\hspace{2mm}}

%\logo{\includegraphics[height=0.4in]{./style_files_stanford/SU_New_BlockStree_2color.png}}

% commands to relax beamer and subfig conflicts
% see here: https://tex.stackexchange.com/questions/426088/texlive-pretest-2018-beamer-and-subfig-collide
\makeatletter
\let\@@magyar@captionfix\relax
\makeatother

\title[]{Towards MDP.jl: The Julia Library of MD Potentials}
%\subtitle{Subtitle Of Presentation}

%\beamertemplatenavigationsymbolsempty

\begin{document}

\author[Julia Lab and PSAAP-3 Team, MIT.]{
	\begin{tabular}{c} 
	%\Large
	%Emmanuel Lujan\\
    %\footnotesize \href{mailto:eljn@mit.edu}{eljn@mit.edu}\\\\
\end{tabular}
\vspace{-4ex}
}

%\institute{
%	\vskip 5pt
%	\begin{figure}
%		\centering
%		\begin{subfigure}[t]{0.5\textwidth}
%			\centering
%			\includegraphics[height=0.60in]{./uba-logo.png}
%		\end{subfigure}%
%		\begin{subfigure}[t]{0.5\textwidth}
%			\centering
%			\includegraphics[height=0.60in]{./mit-logo.jpg}
%		\end{subfigure}
%	\end{figure}
%	\vskip 5pt
%	Departamento de Computación\\
%	Facultad de Ciencias Exactas y Naturales\\
%	Universidad de Buenos Aires\\
%	\vskip 3pt
%}

% \date{June 15, 2020}
\date{\today}

\begin{noheadline}
\begin{frame}\maketitle\end{frame}
\end{noheadline}

\setbeamertemplate{itemize items}[default]
\setbeamertemplate{itemize subitem}[circle]

\begin{frame}
 	\frametitle{Overview} % Table of contents slide, comment this block out to remove it
 	\tableofcontents % Throughout your presentation, if you choose to use \section{} and \subsection{} commands, these will automatically be printed on this slide as an overview of your presentation
 \end{frame}
 
 

%%%%%%%%%%%%%%%%%%%%%%%%%%%%%%%%%%%%%%%%%%%%%%%%%%%%%%%%%%%%%%%%%%%%%%%%%%%%%%%%%%%%%%%%%%%%%%%%%%%%%%%%

\section{MDP.jl: the Julia library of MD potentials}

\begin{frame}
\frametitle{MDP.jl: the Julia library of MD potentials}

\begin{itemize}

\item \textbf{Molecular Dynamics (MD) simulations require a potential energy function} to describe the force field governing the interaction among atoms.
\pause
\item Force calculation often takes between 50\% and 90\% of the overall complexity, thus, \textbf{determining an adequate potential is a crucial task}.
\pause
\item MDP.jl will provide \textbf{fast and accurate potentials for classical MD simulations on exascale supercomputers.}
\begin{itemize}
    \item Coupling empirical and machine learning (ML) potentials
    \item Quantifying uncertainties in trained ML potentials. 
    \item Identifying near-optimal configurations to include in the training data.
\end{itemize}

\end{itemize}
\end{frame}

%%%%%%%%%%%%%%%%%%%%%%%%%%%%%%%%%%%%%%%%%%%%%%%%%%%%%%%%%%%%%%%%%%%%%%%%%%%%%%%%%%%%%%%%%%%%%%%%%%%%%%%%

\begin{frame}
\frametitle{Potential energy fuctions}

\begin{itemize}
\item Potential energy fuctions are determined based on \textbf{quantum information}, e.g. Density Functional Theory (DFT) data.
\pause
\item Classical MD simulations use 
\begin{itemize}
    \item \textbf{Empirical Potentials} (EP), such as Lennard-Jones or Tersoff.
    \item \textbf{Machine Learning Potentials} (MLP), such as Neural Network Potentials (NNP) or Spectral Neighborhood Analysis Potentials (SNAP).
\end{itemize}
\pause
\item \textbf{Present limitations} when solving complex study cases as ultrahigh temperature ceramics in hypersonic flows.
\begin{itemize}
    \item Existing EP, ReaxFF and COMB, do not produce satisfactory results since they \textbf{require retraining.}
    \item MLP \textbf{demand a significant amount of} DFT \textbf{data} for the training process.
\end{itemize}
\end{itemize}
 
\end{frame}
	

%%%%%%%%%%%%%%%%%%%%%%%%%%%%%%%%%%%%%%%%%%%%%%%%%%%%%%%%%%%%%%%%%%%%%%%%%%%%%%%%%%%%%%%%%%%%%%%%%%%%%%%%


\begin{frame}
\frametitle{Coupling Empirical and ML potentials}

\begin{itemize}
\item Coupling both potentials is a promising approach for
	\begin{itemize}
    	\item \textbf{Reducing the amount of training data}
	    \item \textbf{Leveraging physics information coded in the EP}
	\end{itemize}
\pause
\item Optimization problem
\begin{equation*}
\bm c^* = \arg \min_{\bm c \in \mathbb{R}^M } \sum_{j=1}^J w_j \sum_{i=1}^{N_j}  \left|\bm f(\bm {r}^{N_j}, \bm c, j, i) - \bm f^{\rm qm}(\bm r^{N_j}, j, i) \right|^2.
\end{equation*}
\pause
\begin{equation*}
    \bm f(\bm {r}^{N_j}, \bm c, j, i) = 
    \bm f_{E}(\bm {r}^{N_j}, \bm c, j, i) +
    \bm f_{ML}(\bm {r}^{N_j}, \bm c, j, i)
\end{equation*}
\end{itemize}
\end{frame}


%%%%%%%%%%%%%%%%%%%%%%%%%%%%%%%%%%%%%%%%%%%%%%%%%%%%%%%%%%%%%%%%%%%%%%%%%%%%%%%%%%%%%%%%%%%%%%%%%%%%%%%%

\section{Descriptor and force calculation}


\begin{frame}
\frametitle{Empirical component of the force}

Empirical component of the force:
\begin{equation*}
    \bm f_E(\bm {r}^{N_j}, \bm c, j, i) = 
    \bm f_{E,PS}(\bm {r}^{N_j}, \bm c, j, i) +
    \bm f_{E,BS}(\bm {r}^{N_j}, \bm c, j, i)
    \label{eq:force}
\end{equation*}
\pause
The \textbf{power spectrum} component (long range interaction):
\begin{equation*}
    \bm f_{E,PS}(\bm {r}^{N_j}, \bm c, j, i) = 
    \sum_{t=1}^{N_z}
    \sum_{k=1}^K
    \sum_{k'=k}^{K}
    \sum_{l=0}^L
    c_{tkk'l}
    \frac{\partial d^{PS}_{tkk'l}(\bm r^{N_j}, j, i)}{\partial \bm r^{N_j}_i}
    \label{eq:force}
\end{equation*}
\pause
The \textbf{bispectrum} component (short range interactions):
\begin{equation*}
    \bm f_{E,BS}(\bm {r}^{N_j}, \bm c, j, i) = 
    \sum_{t=1}^{N_z}
    \sum_{k=1}^K
    \sum_{k'=k}^{K}
    \sum_{l=0}^L
    \sum_{l_1=0}^L
    \sum_{l_2=0}^L
    c_{tkk'l{l_1}{l_2}}
    \frac{\partial d^{BS}_{tkk'l{l_1}{l_2}}(\bm r^{N_j}, j, i)}{\partial \bm r^{N_j}_i}
    \label{eq:force}
\end{equation*}


\end{frame}
 
 
%%%%%%%%%%%%%%%%%%%%%%%%%%%%%%%%%%%%%%%%%%%%%%%%%%%%%%%%%%%%%%%%%%%%%%%%%%%%%%%%%%%%%%%%%%%%%%%%%%%%%%%%


\begin{frame}
\frametitle{Empirical potential: PS basis functions}

Derivative of the power spectrum basis functions
\begin{equation*}
    \label{eq:derd}
    \frac{\partial d^{BS}_{tkk'l}(\bm r^{N_j}, j, i)}{\partial \bm r^{N_j}_i} = \sum_{s \in \Omega'_{jit}} p_{iskk'l}^{\partial}(\bm r^{N_j}, j) - \sum_{s \in \Omega''_{jit}} p_{sikk'l}^{\partial} ( \bm r^{N_j}, j),
\end{equation*}
\begin{equation*}
    p_{i_0i_1kk'l}^{\partial}(\bm r^{N_j}, j) = \sum_{m=-l}^l \left( \frac{\partial u_{klm}(\bm r_{i_0}^{N_j}- \bm r_{i_1}^{N_j})}{\partial (\bm r_{i_0}^{N_j}-\bm r_{i_1}^{N_j})} \sum_{s \in \Omega_{j,i_1}} \left( u_{k'lm} (\bm r^{N_j}_s- \bm r^{N_j}_{i_1}) \right) \right) + 
\end{equation*}
\begin{equation*}
\sum_{m=-l}^l \left( \frac{\partial u_{k'lm}(\bm r_{i_0}^{N_j}- \bm r_{i_1}^{N_j})}{\partial (\bm r_{i_0}^{N_j}- \bm r_{i_1}^{N_j})} \sum_{s \in \Omega_{j,i_1}} \left( u_{klm} (\bm r^{N_j}_s- \bm r^{N_j}_{i_1}) \right) \right)
\end{equation*}


\end{frame}
 
 
%%%%%%%%%%%%%%%%%%%%%%%%%%%%%%%%%%%%%%%%%%%%%%%%%%%%%%%%%%%%%%%%%%%%%%%%%%%%%%%%%%%%%%%%%%%%%%%%%%%%%%%%

\begin{frame}
\frametitle{Empirical potential: PS basis functions}

\begin{itemize}
\item The descriptors are based on products of radial basis functions $g_{lk}(r)$ and spherical harmonics $Y_{lm}(\theta,\phi)$.
\begin{equation*}
u_{klm}(\bm r) =  g_{lk}(r) Y_{lm}(\theta,\phi)
\end{equation*}
\pause
\item $g_{lk}(r)$ can vary...
\begin{itemize}
    \item Spherical Bessel
    \item Polynomials, Gaussian functions, etc.
\end{itemize}
\pause
\item Providing flexibility when calculating the derivative of $u_{klm}(\bm r)$ using the finite difference method (FDM)
\begin{equation*}
     \frac{\partial u_{klm}(\bm r)}{\partial (\bm r)} =  
        \bigg( \frac{ u_{klm}(\bm r + \bm \Delta x) - u_{klm}(\bm r - \bm \Delta x) } {2 |\bm \Delta x|}, ..., ... \bigg)
\end{equation*}
\end{itemize}


\end{frame} 
 

 
 
%%%%%%%%%%%%%%%%%%%%%%%%%%%%%%%%%%%%%%%%%%%%%%%%%%%%%%%%%%%%%%%%%%%%%%%%%%%%%%%%%%%%%%%%%%%%%%%%%%%%%%%%


\begin{frame}
\frametitle{Empirical potential: BS basis functions}
\small

Derivative of the bispectrum basis functions
\begin{equation*}
     \frac{\partial d^{BS}_{tkk'l{l_1}{l_2}}(\bm r^{N_j}, j, i)}{\partial \bm r^{N_j}_i} =  
 \bigg( \frac{ d_{tkk'l{l_1}{l_2}}(\bm r^{N_j} + \bm \Delta X_i, j, i) - d_{tkk'l{l_1}{l_2}}(\bm r^{N_j} - \bm \Delta X_i, j, i) } {2 |\bm \Delta X_i|}, ..., ... \bigg)
\label{eq:bs}
\end{equation*}
\pause
The bispectrum basis functions are formulated as:
\begin{equation*}
    d^{BS}_{tkk'l{l_1}{l_2}}(\bm r^{N_j}, j, i) = \sum_{s \in \Omega'''_{jt}} b_{skk'l{l_1}{l_2}}(\bm r^{N_j}, j, i)
\label{eq:d_bs}
\end{equation*}
\begin{equation*}
    b_{skk'l{l_1}{l_2}}(\bm r^{N_j}, j, i) = 
     \sum_{m=-l}^l
     \sum_{m_1=-{l_1}}^{l_1}
     \sum_{m_2=-l_2}^{l_2}
     \bar{a}_{sklm}(\bm r^{N_j}, j)
     C_{{m_1}{m_2}m}^{{l_1}{l_2}l}
     a_{sk'{l_1}{m_1}}(\bm r^{N_j}, j)
     a_{sk'{l_2}{m_2}}(\bm r^{N_j}, j)
\end{equation*}

\begin{equation*}
    a_{iklm}(\bm r^{N_j}, j) = \sum_{s \in \Omega_{ji}} u_{klm}(\bm r^{N_j}_s -\bm r^{N_j}_i)
\end{equation*}
\end{frame}


%%%%%%%%%%%%%%%%%%%%%%%%%%%%%%%%%%%%%%%%%%%%%%%%%%%%%%%%%%%%%%%%%%%%%%%%%%%%%%%%%%%%%%%%%%%%%%%%%%%%%%%%

\section{Next steps...}

\begin{frame}
\frametitle{Next steps}
\small
\begin{itemize}
\item Keep my summary of implemented equations up to date \checkmark
\pause
\item Optimization
	\begin{itemize}
   	 \item Clebsch–Gordan coefficients
   	     \begin{itemize}
   	 		\item Fast calculation using ''PartialWaveFunctions.jl`` \checkmark
			\item Stop calculation when a zero is found \checkmark
    	\end{itemize} 
    \item Calculate only in the neighbors
        \begin{itemize}
    		\item Precalculated neighbors information \checkmark
    		\item Optimize precalculation
    	\end{itemize}
	\item Leverage spherical harmonics symetries
	\item Threading and GPU
	\end{itemize}
\pause
\item Testing: the descriptors are invariant to 
	\begin{itemize}
	 \item Rotation \checkmark
   	 \item Permutation
   	 \item Translation
\end{itemize}
\pause
\item ML potentials
\pause
\item Integration with NBodysimulator and LAMMPS
\end{itemize}


\end{frame}


%%%%%%%%%%%%%%%%%%%%%%%%%%%%%%%%%%%%%%%%%%%%%%%%%%%%%%%%%%%%%%%%%%%%%%%%%%%%%%%%%%%%%%%%%%%%%%%%%%%%%%%%


\begin{frame}
\frametitle{MDP.jl + NBodySimulator.jl}
\small
Argon study case.

\begin{enumerate}
    \item Run MDP.jl:
     	\begin{itemize}
    		\item Input (from the initial condition):
	    		\begin{itemize}
    			\item Force field (calculated through LJ, thus, the initial force field). Atomic positions.
	    		\end{itemize}
    		\item Output:
	    		\begin{itemize}
    			\item Force field (this fitted function will be used in the whole MD simulation)
	    		\end{itemize}
    	\end{itemize}
    \item Run NBodySimulator.jl
    
	    For each time step:
		\begin{itemize}
    		\item Input (from the initial condition):
	    		\begin{itemize}
    			\item Force field from MDP.jl (the fitted force field expression is evaluated here). Atomic Velocities. Domain(box). Temperature. Mass. External forces.
	    		\end{itemize}
    		\item Output:
	    		\begin{itemize}
    			\item Atom positions. Atomic velocities. Temperature. Energy (kinetic, potential, and total). Radial distribution function
	    		\end{itemize}
    	\end{itemize}
\end{enumerate}
\end{frame}

\begin{frame}
\frametitle{DFTK.jl + NBodySimulator.jl}
\small
Argon study case.

\begin{enumerate}
    \item Run NBodySimulator.jl
    
	    For each time step:
		\begin{itemize}
    	\item Input: Quantum force from DFTK.jl
	    		\begin{itemize}
    			\item DFTK.jl input: Atomic positions from NBodySimulator.jl
				\item DFTK.jl output: Quantum force (a single set of forces evaluated at one set of positions). Atomic positions. Atomic velocities. Domain (box). Temperature. Mass. External forces
				\end{itemize}
		\item Output: Atom positions. Atomic velocities. Temperature. Energy (kinetic, potential, and total). Radial distribution function.
		\end{itemize}
      
\end{enumerate}
\end{frame}



%%%%%%%%%%%%%%%%%%%%%%%%%%%%%%%%%%%%%%%%%%%%%%%%%%%%%%%%%%%%%%%%%%%%%%%%%%%%%%%%%%%%%%%%%%%%%%%%%%%%%%%%

\begin{frame}
\frametitle{References}
\small
\begin{itemize}
    \item ``MDP.jl: The Julia Library of Molecular
Dynamics Potentials''. MIT PSAAP-3 Team. 2021.    
    \item ``Accelerating Force Calculation in Molecular
Dynamics with Transition to LAMMPS''. MIT PSAAP-3 Team. 2021.  
    \item MDP. \url{https://github.com/cesmix-mit/MDP.jl}
    \item Fortran and Ar MD simulation. \url{https://ase.tufts.edu/chemistry/lin/outreach_FortranMD.html}
    \item NBodySimulator. \url{https://github.com/SciML/NBodySimulator.jl}
    \item Molly. \url{https://github.com/JuliaMolSim/Molly.jl}    
    \item Argon study case using NBodySimulator. \url{https://github.com/jrdegreeff/cesmix-julia/blob/main/notebooks/nbodysimulator_argon_simulation.jl}
    \item Draft DFTK+Molly / MDP+Molly. \url{https://docs.google.com/document/d/1APQ1JjL0SoQTTsDam5j3lEKe3xCDwNzrbRIGNH_JzxU/edit?usp=sharing}
\end{itemize}

\end{frame}

%%%%%%%%%%%%%%%%%%%%%%%%%%%%%%%%%%%%%%%%%%%%%%%%%%%%%%%%%%%%%%%%%%%%%%%%%%%%%%%%%%%%%%%%%%%%%%%%%%%%%%%%

\begin{frame}
\frametitle{}

\centering\textbf{Thanks :-)}
\end{frame}

%%%%%%%%%%%%%%%%%%%%%%%%%%%%%%%%%%%%%%%%%%%%%%%%%%%%%%%%%%%%%%%%%%%%%%%%%%%%%%%%%%%%%%%%%%%%%%%%%%%%%%%%


%\begin{frame}
% \frametitle{Computational Science: Challenges}

% Software tools
% \begin{itemize}
% \item Private Software Packages
%         \begin{itemize}
%     	\item COMSOL, ANSyS Fluent
%     	\end{itemize}
% \item Open Source Packages
%         \begin{itemize}
%     	\item OpenFoam, Matlab
%     	\end{itemize}
% \end{itemize}

% \end{frame}




% It would be good to get something merged here. I think some of the internals need to converge with @KirillZubov 's PINN extensions https://nextjournal.com/kirill_zubov/physics-informed-neural-networks-pinns-solver-on-julia-gsoc-2020-final-report so this project is falling a bit behind. Also, it needs to start taking into account the boundary changes coming in SciML/ModelingToolkit.jl#526

%I was thinking that it might be good to get a publication together for DiffEqOperators. It's almost a 1000 commit repo now where most of the work was done by different students over the years, so I think all it needs is someone to really finish up the project and it's an easy paper. If you wanted to take that and be first author I think it could be an easy 6 months paper. It can highlight some of the stuff like automatic GPU capability with cudnn handling, multithreading, etc. The last step to finishing the repo is getting the multidimensional BCs working (which is something you might run into in the next PR). So feel free to take it to first author. 

%I think DiffEqOperators may need to be finished first, since the auto-discretization is going to rely on those operators, which will have an issue with the higher dimensional boundary conditions. FWIW it should only be a month or two of work, and I think Dan should get hired soon and he should be helping finish it as well.



% ------

% in order to handle "all" PDEs... we have to go symbolic
% enter ModelingToolkit.jl
% ModelingToolkit.jl: an open source compiler + IR for Models and Transformations 
%     automated acceleration of numerical methods
%         generation of jacobian and hessians
%         model order reduction
%         automated heterogeneus paralellism
%     transformation into "solvable forms"
%         pantelides algorithm
%         lamperti transform
%     target other compilation
%         task-based decomposition
%         embedded outputs
%     define partial differential equations
    
% A structured and documented IR for describing models
% "Model transforms" are compiler passes IR -> IR
% Extendable high level "context" system
% Let other write domain specific languages on top
%     DSLs should not have to define simplification, differentiation, etc
    
% ModelToolkit is a symbolic framework where you can define your derivative terms, your parameters, you define an ODE system, and then you can do transformation on top of it. It is build to do automatic code optimization, and handle differential algebraic equations.
% You can use this symbolic system to represent PDEs

% Typically you take your derivatives, discretized them and the at the bottom you end up with a loop. But this means that you can reuse your code. 

% Method of Lines Finite Difference: you tell the which dimension is your time dimension and you are gonna discretize everything except your time dimension, and it gives you out an ODE problem. Then you solve the problem with DifferentialEquation.jl

% The porpuse of discretization is not to solve the problem but is to give you a solvable problem

% Here and abstract description of the PDE system


% - Mathematics is the language of science
% - Julia is the new mathematics

% Ergo, Julia is the new language of science

% Toda ciencia tiene de ciencia lo que tiene de matemáticas Poincaré (1854 1912)

% - Mathematical modeling is a fundamental tool for describing, predicting and optimizing complex phenomena.

% mathematical modeling is a fundamental tool for describing, predicting and optimizing EP-based methodologies.


% \section{Introduction}
% % `` can be used to have multiple slides in one frame, where the slides are continued with the suffix "(cont.)"; `` can be used with `\framebreak` to manually break each frame into multiple slides
% \begin{frame}
% \frametitle{Introduction}
% 	Itemize example
% 	\begin{itemize}
% 		\item Item 1
% 		\item Item 2
%         \begin{table}[]
%         \caption{Example of Table - Taxonomy of human intent prediction}
%         \label{tab:table_example}
%         \vspace{-.75cm}
%         \resizebox{0.95\textwidth}{!}{%
%         \begin{tabular}{|c|c|c|c|}
%         \hline
%         \multicolumn{2}{|c|}{\multirow{2}{*}{Human}} & \multicolumn{2}{c|}{\begin{tabular}[c]{@{}c@{}}Execution Strategy\\ (Action)\end{tabular}}                            \\ \cline{3-4} 
%         \multicolumn{2}{|c|}{}                       & \begin{tabular}[c]{@{}c@{}}Observer\\ Knows\end{tabular} & \begin{tabular}[c]{@{}c@{}}Observer\\ Unknown\end{tabular} \\ \hline
%         \multirow{2}{*}{\begin{tabular}[c]{@{}c@{}}Objective \\ Function\end{tabular}} &
%           \begin{tabular}[c]{@{}c@{}}Observer\\ Knows\end{tabular} &
%           \begin{tabular}[c]{@{}c@{}}All is Known (e.g. Ping Pong) \\ where both objective and actions are clear\end{tabular} &
%           \begin{tabular}[c]{@{}c@{}}Human Action Model is unclear\\  or suboptimal (e.g. chess)\end{tabular} \\ \cline{2-4} 
%          &
%           \begin{tabular}[c]{@{}c@{}}Observer\\ Unknown\end{tabular} &
%           \begin{tabular}[c]{@{}c@{}}Human action model is well known, \\ but objective is not (e.g. joy-riding in car \\ or free running, where destination\\  or direction is unclear)\end{tabular} &
%           \begin{tabular}[c]{@{}c@{}}Poor action model and objective\\  function (e.g. Poor / good cook, \\ no idea of final dish)\end{tabular} \\ \hline
%         \end{tabular}%
%         }
%         \end{table}

% 		\item Tables can be referenced as Table  \ref{tab:table_example}
% 	\end{itemize}
	
% 	\framebreak
	
% 	Example of a figure, shown in Figure \ref{fig:prob_formulation_scenario_1}.
	
% 	\begin{figure}
%         \centering
%         \includegraphics[width=0.8\textwidth]{images/prob_formulation_scenario_1.png}
%         \caption{Example Figure}
%         \label{fig:prob_formulation_scenario_1}
%     \end{figure}
% \end{frame}

% % This demonstrates a new section
% \section{Examples}
% % This demonstrates a single frame without framebreaks
% \begin{frame}{Example of Horizontal Subfigures}

% 	\begin{figure}
% 		\centering
% 		\begin{subfigure}[t]{0.5\textwidth}
% 			\centering
% 			\includegraphics[width=0.9\textwidth]{images/stone2014fall_setup.png}
% 			\caption{Single Kinect setup for fall prevention in elderly residence \cite{stone2014fall}}
% 		\end{subfigure}%
% 		~ 
% 		\begin{subfigure}[t]{0.5\textwidth}
% 			\centering
% 			\includegraphics[width=\textwidth]{images/staranowicz2015easy_multiple_kinects.png}
% 			\caption{Multiple Kinects calibration for fall prediction\cite{staranowicz2015easy}}
% 		\end{subfigure}
% 		\caption{Examples of Horizontal Subfigures}
% 	\end{figure}
% \end{frame}

% \begin{frame}{Example of Horizontal Alignment}
%     % For data collection:
    
%     Example of Horizontal Alignment of a \texttt{table} and a \texttt{figure}.
%     \begin{center}
%     \begin{minipage}[t]{.65\linewidth}
%     \begin{table}[H]
%     % \renewcommand{\arraystretch}{1.3}
%     \caption{Environment limitations on data collection}
%     \label{tab:env_limit}
%     \centering
%     % \begin{tabular}{m{1.6cm}|c|>{\centering\arraybackslash}m{2cm}|>{\centering\arraybackslash}m{2.3cm}}
%     \begin{tabular}{m{2cm}|c|c|>{\centering\arraybackslash}m{1.5cm}}
%     % \begin{tabular}{c|c|c|c}
%         & Kinect & Stereo & Kinect + Stereo\\
%         \hline
%         Indoor & \cmark & \cmark & \cmark \\
%         \hline
%         Outdoor & \xmark & \cmark & \cmark \\
%         \hline
%         High number of features & \cmark & \cmark & \cmark \\
%         \hline
%         Low number of features & \cmark & \xmark & \cmark 
%     \end{tabular}
%     \end{table}
%     \end{minipage}%
%     \begin{minipage}[t]{.35\linewidth}
%     \vspace{0pt}
%     \centering
%     \includegraphics[width=0.7\textwidth]{images/waist_cam_setup_new.png}
%     \end{minipage}
%     \end{center}
% \end{frame}

% \begin{frame}
% \frametitle{Example of resizable equations}

% \begin{center}
% \scalebox{1.0}{\parbox{\linewidth}{%
% 		\begin{align*}
% 		& {\text{min \hskip 6pt}}
% 		& & J = \int (a_{real} - \hat{a})^2  \\
% 		& \text{subject to}
% 		& & \text{human kinematics} \\
% 		&&& \text{no collision} \\
% 		&&& \text{no falling} 
% 		\end{align*}
% }}
% \end{center}
% \end{frame}

% \begin{frame}
% \frametitle{Example of Regular Equations}
%     % \begin{equation}
%     %     {}^Ag = {}^AR_B {}^Bg
%     % \end{equation}
    
%     % \begin{equation}
%     %     V = \frac{{}^Bg \cross {}^Ag}{\norm{{}^Ag}\norm{{}^Bg}}, 
%     %     \theta = \arccos{\frac{{}^Bg \cross {}^Ag}{\norm{{}^Ag}\norm{{}^Bg}}}
%     % \end{equation}
    
%     \begin{equation}
%         \begin{split}
%         {}^AR_{B}(t_0)=\left[\begin{array}{ccc}
%         1 & 0 & 0 \\
%         0 & 1 & 0 \\
%         0 & 0 & 0
%         \end{array}\right]+
%         \sin (\theta)\left[\begin{array}{ccc}
%         0 & -v_{3} & v_{2} \\
%         v_{3} & 0 & -y_{1} \\
%         -v_{2} & v_{1} & 0
%         \end{array}\right]+ \\
%         (1-\cos (\theta))\left[\begin{array}{ccc}
%         0 & -v_{3} & v_{2} \\
%         v_{3} & 0 & -v_{1} \\
%         -v_{2} & v_{1} & 0
%         \end{array}\right]^{2}
%         \end{split}
%         \end{equation}
        
%         \begin{align}
%             {}^AR_{B}(t) &= \Delta R {}^AR_{B}(t_0) \\
%             \Delta R &= {}^AR_{B}(t) {}^AR_{B}^T(t_0)
%         \end{align}
% \end{frame}

% \begin{frame}
% \frametitle{Example of Video}

% 	\includemedia[
% 	width=\linewidth,
% 	totalheight=0.6\linewidth,
% 	activate=pageopen,
% 	passcontext,  %show VPlayer's right-click menu
% 	addresource=images/opensim_video.mp4,
% 	flashvars={
% 		%important: same path as in `addresource'
% 		source=images/opensim_video.mp4
% 	}
% 	]{\fbox{Click!}}{VPlayer.swf}
    
% \end{frame}

% \begin{frame}
% \frametitle{Bibliography}
% \printbibliography
% \end{frame}

\end{document}