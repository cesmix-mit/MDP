\documentclass[12pt]{article}
\usepackage{fullpage}%varioref,multicol

%\topmargin 0.25in
\usepackage{bm}
%\usepackage{latexsym}
\usepackage{amsmath,amssymb}
%\usepackage{amssymb,amsbsy,exscale,amsmath,amsfonts,amssymb,amscd}
\usepackage{amsthm}
\usepackage{wrapfig}
\usepackage{epsfig}
\usepackage{graphicx}
\usepackage{makeidx}
\usepackage{multicol}
\usepackage{tikz}
\usepackage{algorithm}
\usepackage{algpseudocode}
\usepackage{soul}

%\usepackage{subeqn}
%\usepackage{crop}
%\crop
\newcommand{\pr}{\mathrm{pr}}
\newcommand{\du}{\mathrm{du}}
\newcommand{\RR}{{\mathbb R}}
\newcommand{\K}{{\mathcal K}}
\newcommand{\T}{{\mathbb T}}
\newcommand{\Em}{{\mbox{\small \,${\rm E}$\,--\,}}}
\newcommand{\Ep}{{\mbox{\small \,${\rm E}$\,+\,}}}
\newcommand{\N}{{\cal N}}
\newcommand{\M}{{\cal M}}
\newcommand{\D}{{\cal D}}
\newcommand{\I}{{\mathbb I}}
\newcommand{\refeq}[1]{(\ref{#1})}
\newcommand{\R}{I\hspace{-1ex}R}
\newtheorem{thm}{Theorem}
\newtheorem{lem}[thm]{Lemma}
\newcommand{\ex}{{\rm ex}}
\newcommand{\bk}{{\rm bk}}
\newcommand{\df}{{\rm df}}
\newcommand{\std}{{\rm std}}
\newcommand{\cov}{{\rm cov}}
\newcommand{\true}{{\rm true}}
%\makeindex


% definitions used by included articles, reproduced here for
% educational benefit, and to minimize alterations needed to be made
% in developing this sample file.


%%%%%%%%%%%%%%%%%%%%%%%%%%%%%%%%%%%%%%%%%%%%%%%%%%%%%%%%%%%%%%%%%%%%%%%%
\graphicspath{{../../figures/}}
%\DeclareGraphicsExtensions{.jpg,.png,.mps,.pdf}
\DeclareGraphicsRule{*}{mps}{*}{}
%%%%%%%%%%%%%%%%%%%%%%%%%%%%%%%%%%%%%%%%%%%%%%%%%%%%%%%%%%%%%%%%%%%%%%%%

%Define shortcuts
\newcommand{\del}[2]{\frac{\partial #1}{\partial #2}}
\newcommand{\ddel}[2]{\dfrac{\partial #1}{\partial #2}}
\def\mathbi#1{\textbf{\em #1}}

\newcommand\rev[1]{\textcolor{black}{#1}}
\newcommand\revisePF[1]{\textcolor{red}{#1}}

\newcommand{\bs}[1]{\boldsymbol{#1}}
\newcommand{\bmu}{\bs{\mu}}

\usepackage{tabularx}
\usepackage{booktabs}


\begin{document}


\title{GOLFF.jl: Generation and Optimization of Machine
Learning Force Field - Summary of the Implemented Equations - Draft}

\date{\today}

%
\maketitle

\begin{abstract}
This draft presents a summary of the equations that are currently implemented in GOLFF.jl (and its auxiliary files). It is based on the file GOLFF.pdf. 
Some of the terminology was slightly redefined in order to smooth the transition between the original equations and their computational implementation. This draft was not yet validated.
\end{abstract}

\section{Input Variables/Parameters}

The following table shows all the relevant input variables or parameters needed in the computational implementation. As mentioned above, some of them were slightly redefined and new definitions were added. 

\begin{center}
    \label{table:def}
    \begin{tabular}{ | l | p{7.5cm} | l | p{7.5cm} |}
    \hline
    $J$ & Number of configurations. & $\Omega_{ji}$ & Neighbors of the atom $i$ in configuration $j$. \\ \hline
    
    $N_j$ & Number of atoms in configuration $j$. & $\Omega'_{jit}$ & Neighbors of the atom $i$ in configuration $j$, whose atomic number type is $t$. \\ \hline
    
    $Z_{ji}$ & Atomic number of atom $i$ in configuration $j$. & $\Omega''_{jit}$ & If the atomic number type of $i$ is $t$, returns the neighbors of the atom $i$ in configuration $j$, else it returns empty. \\ \hline
    
    $T_z$ & Type of the atomic number z. Each atomic number z is indexed in T. & $K$ & ? \\ \hline
    
    $N_z$ & Number of the different atomic numbers (or atomic number types) present in all configurations. &  L & Degree \\ \hline
    
    $\bm {r}^{N_j}_{i}$ & Position of atom $i$ in the configuration $j$. & M & Number of basis functions.\\ \hline
    
    $\bm {r}^{N_j}$ & Positions of all the atoms in configuration $j$ ($N_j \times 3$). & $c_{tkk'l}$  & Coefficient needed to calculate the potential/force. A linearized version $c_{m}$ was finally used. See Equation \ref{eq:force}.\\ \hline %[1ex]
    
    $\bm f^{\rm qm}_{ji}$ & Quantum force associated to the atom $i$ in the configuration $j$. & $w_j$ & Weight associated to the configuration j. \\ \hline
    
    $r_{\rm cut}$ & Cut radius needed to calculate the neighbors of each atom. &  $h$ & Finite difference increment. \\
    \hline
    \end{tabular}
\end{center}

\section{Equation Summary}
\subsection{Main goal}

The goal is to find an optimal coefficient vector $\bm c^*$ to match a set of input quantum mechanical (QM) forces. The optimization problem is formulated in the Equation 4 of the original manuscript. Here, the problem is formulated as follows:

\begin{equation}
\bm c^* = \arg \min_{\bm c \in \mathbb{R}^M } \sum_{j=1}^J w_j \sum_{i=1}^{N_j}  \left|\bm f_{ji}(\bm {r}^{N_j}, \bm c) - \bm f^{\rm qm}_{ji}(\bm r^{N_j}) \right|^2.
\label{eq:min1}
\end{equation}

where $w_j$ are the user-defined weights in configuration $j$, $1 \le j \le J$, the default value of $w_j$ is 1. $\bm f_{ji}$  depicts a (complex) atomic force, the definition of which can be found below. The value of this force depends on the configuration $j$, the atom $i$ in the $j$-th configuration, the position of all atoms in the $j$-th configuration $\bm {r}^{N_j}$, and the variable for which the minimization is performed $\bm c \in \mathbb{R}^M$. An analogous description is associated to the input QM forces $\bm f^{\rm qm}_{ji}$, with the exception of the dependence of variable $\bm c$.

Each atomic force $\bm f_{ji}$ is calculated through a set of basis functions, namely $d_{tkk'l}(\bm r^{N_j})$. In the original manuscript this force is defined in Equations 4 and 5. Here, the atomic force is formulated as follows:

\begin{equation}
    \bm f_{ji}(\bm {r}^{N_j}; \bm c) = 
    \sum_{t=1}^{N_z}
    \sum_{k=1}^K
    \sum_{k'=k}^{K}
    \sum_{l=0}^L
    c_{tkk'l}
    \frac{\partial d_{tkk'l}(\bm r^{N_j})}{\partial \bm r^{N_j}_i}
    \label{eq:force}
\end{equation}

Note that $ \frac{\partial d_{tkk'l}(\bm r^{N^j})}{\partial \bm r_i}$ is the gradient vector of $d_{tkk'l}$ with respect to the position of the $i$-th atom in the $j$-configuration, i.e., $f$ is a vector quantity.

Current Julia implementation uses ``GalacticOptim'' package to perform the optimization. In the computational implementation $c_{tkk'l}$ had to be linearized due to interface compatibility with the optimization library. The new coefficient is $c_{m}$, where the index $m$ is just a one-dimension unrolling of the tuple index $t$,$k$,$k'$,$l$.

In future versions, neural networks will be used to perform the optimization.

\subsection{Basis functions}

The derivatives of the basis functions are defined in Equations 23 and 24 in the original manuscript. Here, it is formulated as:

\begin{equation}
    \label{eq:derd}
    \frac{\partial d_{tkk'l}(\bm r^{N_j})}{\partial \bm r^{N_j}_i} = \sum_{s \in \Omega'_{jit}} p_{iskk'l}^{\partial}(\bm r^{N_j}, j) - \sum_{s \in \Omega''_{jit}} p_{sikk'l}^{\partial} ( \bm r^{N_j}, j),
\end{equation}
where 
\begin{equation}
    p_{i_0i_1kk'l}^{\partial}(\bm r^{N_j}, j) = \sum_{m=-l}^l \left( \frac{\partial u_{klm}(\bm r_{i_0}^{N_j}- \bm r_{i_1}^{N_j})}{\partial (\bm r_{i_0}^{N_j}-\bm r_{i_1}^{N_j})} \sum_{s \in \Omega_{j,i_1}} \left( u_{k'lm} (\bm r^{N_j}_s- \bm r^{N_j}_{i_1}) \right) \right) + 
\end{equation}
\begin{equation*}
\sum_{m=-l}^l \left( \frac{\partial u_{k'lm}(\bm r_{i_0}^{N_j}- \bm r_{i_1}^{N_j})}{\partial (\bm r_{i_0}^{N_j}- \bm r_{i_1}^{N_j})} \sum_{s \in \Omega_{j,i_1}} \left( u_{klm} (\bm r^{N_j}_s- \bm r^{N_j}_{i_1}) \right) \right)
\end{equation*}

$\Omega$, $\Omega'$, and $\Omega''$ were introduced to simplify the notation and to have precomputed neighbor information. Two atoms, $s$ and $q$, are neighbours if $s \ne q$ and $|| \bm r_s^{N_j} - \bm r_q^{N_j} || \le r_{\rm cut} $ \textbf{but this must take into account the periodic boundary conditions}. 

The derivative of the (complex) function $u_{klm}$ is not present in the original manuscript. Here, it is computed through the finite difference method.

\begin{equation}
\begin{split}
     \frac{\partial u_{klm}(\bm r)}{\partial (\bm r)} =  
        \bigg( \frac{ u_{klm}(\bm r + \bm \Delta x) - u_{klm}(\bm r - \bm \Delta x) } {2 |\bm \Delta x|}, \\
          \frac{ u_{klm}(\bm r + \bm \Delta y) - u_{klm}(\bm r - \bm \Delta y) } {2 |\bm \Delta y|},\\
          \frac{ u_{klm}(\bm r + \bm \Delta z) - u_{klm}(\bm r - \bm \Delta z) } {2 |\bm \Delta z|} \bigg)
\end{split}
\end{equation}

where $\bm \Delta x = (h, 0, 0)$, $\bm \Delta y = (0, h, 0)$, and $\bm \Delta z = (0, 0, h)$.

The function $u_{klm}: \mathbb{R}^3 \rightarrow \mathbb{C}$ is defined in the Equation 11. Here, it is formulated as

\begin{equation}
    \label{eq:us}
    u_{klm}(\bm r) =  g_{lk}(r) Y_{lm}(\theta,\phi)
\end{equation}

Note that $\bm r$ is expressed in cartesian coordinates $(x,y,z)$, but spherical coordinates are needed to invoke $g_{lk}$ and $Y_{lm}$. Then, the following transformations are used:

\begin{equation*}
    r = \sqrt{x^2 + y^2 + z^2}
\end{equation*}
\begin{equation*}
    \theta = \arccos(z/r)    
\end{equation*}
\begin{equation*}
    \phi = \text{atan}(y,x) \footnote{Use the Julia atan function with two arguments ($y$,$x$) that returns the angle in radians between the positive $x$-axis and the point $(x, y)$ in the interval $[-\pi, \pi]$.}
\end{equation*}

In Equation \ref{eq:us}, $g_{lk}$ is a radial basis functions. Currently, the Julia spherical Bessel function "sphericalbessely(l, r)" is used. Other function should be implemented in the future, including polynomials and Gaussian functions.

Returning to Equation \ref{eq:us}, the spherical harmonics of degree $l$ and order $m$ ($Y_{lm}$) is presented in Equation 12 in the original manuscript. Here, it is formulated as
\begin{equation}
 Y_{lm}(\theta,\phi) = \sqrt{\frac{(2l + 1)}{4 \pi} \frac{(l-m)!}{(l+m)!}} P_{lm}(\cos(\theta)) e^{i m \phi}
\end{equation}
$P_{lm}$ are the associated Legendre polynomials. They are not included in the original manuscript. Here they are defined as
\begin{equation}
    \label{eq:leg}
    P_{lm}(x) = (-1)^m (1-x^2)^{m/2}  \frac{\mathrm{d}^m}{\mathrm{d}x^m} P_l(x), \ if \ m \ge 0
\end{equation}
\begin{equation}
    \label{eq:leg2}
    P_{lm} (x) = (-1)^{|m|} \frac{(l-|m|)!}{(l+|m|)!} P_{l|m|}, \ if \ m < 0
\end{equation}
$P_l$ is the Legendre polynomial of degree $l$ given by:
\begin{equation}
    \label{eq:leg3}
    P_l(x)=\frac{1}{2^l l!} \frac{\mathrm{d}^l}{\mathrm{d}x^l}(x^2-1)^l
\end{equation}

By combining Equations \ref{eq:leg} and \ref{eq:leg3}, we obtain the following expressions for $P_{lm}(x)$, $l \ge 0$, $0 \le m \le l$ (Equations \ref{eq:leg4} and \ref{eq:leg5}). The case $-l \le m <0$ follows from Equation \ref{eq:leg2}. See Appendix \ref{app:legendre} for a brief proof.

\begin{equation}
    \label{eq:leg4}
    P_{lm}(x) = (-1)^m (1-x^2)^{m/2} \Bigg( \frac{1}{2!} \sum_{k=\lceil \frac{l+m}{2} \rceil}^l  (-1)^{l-k} {l \choose k} {2k \choose l} \frac{(2k-l)!}{(2k-l-m)!} x^{2k-l-m} \Bigg), \quad m >0
\end{equation}
\begin{equation}
    \label{eq:leg5}
    P_{lm}(x) = P_l(x)  \quad m = 0.
\end{equation} 
In Equation \ref{eq:leg4}, $\lceil . \rceil$ denotes the ceiling function.


%The basis functions $\{d_m(\bm r^N), \ 1 \le m \le M \}$ are given by\footnote{The arguments of the basis functions in Equation \ref{eq:min2} are referred to the atoms in the $j$-th configuration.}:
%\begin{equation*}
%    d_{tkk'l}(\bm r^{N_j}) =  \sum_{s, Z_s=t} \sum_{m=-l}^l \left( \left(\sum_{q \in \Omega_{js}} u_{klm}(\bm r_{qs}) \right) \left( \sum_{s \in \Omega_{js}} u_{k'lm} \left(\bm r_{qs} \right)  \right) \right),  \quad t=1, \ldots, NZ,
%\end{equation*}
%where $\bm r_{qs} = \bm r_q^{N_j}- \bm r_s^{N_j}$ (note that $s$ and $q$ are the $s$-th and $q$-th atoms in the configuration $j$). Hereafter, we drop the superscript $N_j$. We remark that there are $M=(NZ (L+1) K (K+1))/2$ basis functions.
%\textbf{Por el momento no sé si s tiene que ser de la confugiración j, entiendo que sí para que tenga sentido hacer la resta teniendo $r^{N_j}$. Tal vez se pueda quitar esta definición.} 



\begin{appendix}
\section{Associated Legendre polynomials}
\label{app:legendre}
    
In this Appendix, it is presented a derivation of Equation \ref{eq:leg4} from the definitions of Legendre polynomials (Equation \ref{eq:leg3}) and associated Legendre polynomials (Equation \ref{eq:leg}).

By applying the binomial theorem:
\begin{equation}
    \label{eq:fleg}
    P_l(x)= \frac{1}{2^l l!} \frac{\mathrm{d}^l}{\mathrm{d}x^l}(x^2-1)^l = \frac{1}{2^l l!} \sum_{k=0}^l {l \choose k} (-1)^{l-k} \frac{\mathrm{d}^l}{\mathrm{d}x^l}(x^{2k}).
\end{equation}

The $l$-th derivative of $x^{2k}$ can be written as:

\[ \frac{\mathrm{d}^l}{\mathrm{d}x^l}(x^{2k}) =
  \begin{cases}
    2k (2k-1) \ldots (2k-l+1) x^{2k-l}      & \quad \text{if } 2k \ge l \\
    0 & \quad \text{if } 2k < l
  \end{cases}
\]

Therefore, replacing the derivative in Equation \ref{eq:fleg}:
\begin{equation*}
    P_l(x) = \frac{1}{2^l} \sum_{k=\lceil l/2 \rceil}^l {l \choose k} (-1)^{l-k} \frac{(2k)!}{(2k-l)! l!} x^{2k-l}
\end{equation*}
\begin{equation}
    \label{eq:fleg2}
    P_l(x) = \frac{1}{2^l} \sum_{k=\lceil l/2 \rceil}^l {l \choose k} (-1)^{l-k} {2k \choose l} x^{2k-l}
\end{equation}

We proceed in a similar way to obtain the $m$-th derivative of $P_l(x)$ for $0 < m \le l$ using Equation \ref{eq:fleg2}:

\[ \frac{\mathrm{d}^m}{\mathrm{d}x^m}(x^{2k-l}) =
  \begin{cases}
    (2k-l) (2k-l-1) \ldots (2k-l-m+1) x^{2k-l-m}      & \quad \text{if } 2k-l \ge m \\
    0 & \quad \text{if } 2k -l < m
  \end{cases}
\]

Consequently:
\begin{equation*}
    P_{lm}(x)=(-1)^m (1-x^2)^{m/2} \frac{\mathrm{d}^m}{\mathrm{d}x^m} P_l(x) = 
\end{equation*}
\begin{equation*}
     = (-1)^m (1-x^2)^{m/2} \frac{1}{2^l} \sum_{k = \lceil \frac{l+m}{2} \rceil}^{l} (-1)^{l-k} {l \choose k} {2k \choose l} \frac{(2k-l)!}{(2k-l-m)!} x^{2k-l-m}
\end{equation*}

\end{appendix}


\end{document}
